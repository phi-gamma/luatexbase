% \iffalse meta-comment
%
% Copyright (C) 2009 by Elie Roux <elie.roux@telecom-bretagne.eu>
%
% This work is under the CC0 license.
%
% This work consists of the main source file luatexbase-mcb.dtx
% and the derived files
%    luatexbase-mcb.sty, mcb.lua, luatexbase-mcb.pdf,
%    test-mcb-plain.tex test-mcb-latex.tex
%
% Unpacking:
%    tex luatexbase-mcb.dtx
% Documentation:
%    pdflatex luatexbase-mcb.dtx
%
%<*ignore>
\begingroup
  \def\x{LaTeX2e}%
\expandafter\endgroup
\ifcase 0\ifx\install y1\fi\expandafter
         \ifx\csname processbatchFile\endcsname\relax\else1\fi
         \ifx\fmtname\x\else 1\fi\relax
\else\csname fi\endcsname
%</ignore>
%<*install>
\input docstrip.tex

\keepsilent
\askforoverwritefalse

\let\MetaPrefix\relax

\preamble

Copyright (C) 2009 by Elie Roux <elie.roux@telecom-bretagne.eu>

This work is under the CC0 license.
See source file '\inFileName' for details.

\endpreamble

\let\MetaPrefix\DoubleperCent

\generate{%
  \usedir{doc/luatex/luatexbase}%
  \file{test-mcb-plain.tex}{\from{luatexbase-mcb.dtx}{testplain}}%
  \file{test-mcb-latex.tex}{\from{luatexbase-mcb.dtx}{testlatex}}%
}

\def\MetaPrefix{-- }

\def\luapostamble{%
  \MetaPrefix^^J%
  \MetaPrefix\space End of File `\outFileName'.%
}

\def\currentpostamble{\luapostamble}%

\generate{%
  \usedir{tex/luatex/luatexbase}%
  \file{mcb.lua}{\from{luatexbase-mcb.dtx}{lua}}%
}

\obeyspaces
\Msg{************************************************************************}
\Msg{*}
\Msg{* To finish the installation you have to move the following}
\Msg{* files into a directory searched by TeX:}
\Msg{*}
\Msg{*     luatexbase-mcb.sty mcb.lua}
\Msg{*}
\Msg{* Happy TeXing!}
\Msg{*}
\Msg{************************************************************************}

\endbatchfile
%</install>
%<*ignore>
\fi
%</ignore>
%<*driver>
\documentclass{ltxdoc}
% preamble for dtx documentations of the luatexbase package/bundle.

% packages
\usepackage[T1]{fontenc}
\usepackage{textcomp}
\usepackage{lmodern}
\usepackage{xspace}
\usepackage[a4paper]{geometry}
\usepackage[english]{babel}
\usepackage[colorlinks]{hyperref}
\usepackage{bookmark}

% logos
\makeatletter
\newcommand \eTeX       {$\m@th\varepsilon$-\TeX}
\newcommand \LuaTeX     {Lua\TeX}
\renewcommand \PlainTeX {Plain\thinspace\TeX}
\newcommand \TeXe       {\TeX\thinspace82}
\newcommand \TeXLive    {\TeX\thinspace Live}
\makeatother

% logos for the lazy me (mpg)
\newcommand \tex      {\TeX\xspace}
\newcommand \latex    {\LaTeX\xspace}
\newcommand \etex     {\eTeX\xspace}
\newcommand \luatex   {\LuaTeX\xspace}
\newcommand \texe     {\TeXe\xspace}
\newcommand \texlive  {\TeXLive\xspace}
\newcommand \plaintex {\PlainTeX\xspace}

% special elements of text
\newcommand*\email [1] {\href{mailto:#1}{#1}}
\newcommand \file       {\nolinkurl}
\newcommand \pk         {\textsf}
\newcommand \cmdname    {\texttt}

% compact version of itemize
\newenvironment{itemize*}{%
  \itemize
  \setlength\itemsep{0pt}%
  \setlength\parskip{0pt}%
}{%
  \enditemize
}

% no so nice, but anyway...
\newenvironment{qcode}{%
  \quote \ttfamily \catcode`\_=12\relax \obeylines
}{%
  \endquote
}

% for hyperref
\pdfstringdefDisableCommands{%
  \def\cs#1{\@backslashchar #1}%
}

% constant metadata
\author{%
  Manuel P\'egouri\'e-Gonnard \& \'Elie Roux \\
  Support: \email{lualatex-dev@tug.org}%
}
\hypersetup{%
  pdfauthor = {Manuel Pegourie-Gonnard and Elie Roux},
}

% other metadata
\newcommand*\pkdate [2] {%
  \title{The \pk{#1} package}
  \date{#2}
  \hypersetup{pdftitle={The #1 package}}
}


% ltxdoc setup
\DisableCrossrefs

\begin{document}
  \DocInput{luatexbase-mcb.dtx}%
\end{document}
%</driver>
% \fi
%
% \CheckSum{0}
%
% \CharacterTable
%  {Upper-case    \A\B\C\D\E\F\G\H\I\J\K\L\M\N\O\P\Q\R\S\T\U\V\W\X\Y\Z
%   Lower-case    \a\b\c\d\e\f\g\h\i\j\k\l\m\n\o\p\q\r\s\t\u\v\w\x\y\z
%   Digits        \0\1\2\3\4\5\6\7\8\9
%   Exclamation   \!     Double quote  \"     Hash (number) \#
%   Dollar        \$     Percent       \%     Ampersand     \&
%   Acute accent  \'     Left paren    \(     Right paren   \)
%   Asterisk      \*     Plus          \+     Comma         \,
%   Minus         \-     Point         \.     Solidus       \/
%   Colon         \:     Semicolon     \;     Less than     \<
%   Equals        \=     Greater than  \>     Question mark \?
%   Commercial at \@     Left bracket  \[     Backslash     \\
%   Right bracket \]     Circumflex    \^     Underscore    \_
%   Grave accent  \`     Left brace    \{     Vertical bar  \|
%   Right brace   \}     Tilde         \~}
%
% \title{The \textsf{luatexbase-mcb} package}
% \date{2009/09/18 v0.93}
% \author{Elie Roux \\ \texttt{elie.roux@telecom-bretagne.eu}}
%
% \maketitle
%
% \begin{abstract}
% This package manages the callback adding and removing, by adding
% \texttt{callback.add} and \texttt{callback.remove}, and overwriting
% \texttt{callback.register}. It also allows to create and call new callbacks.
% For an introduction on this package (among others), please refer to the
% document \texttt{luatextra-reference.pdf}.
% \par\textbf{Warning.} Currently assumes that \textsf{luatexbase-modutils}
% has been previously loaded. (This is a temporary limitation.)
% \end{abstract}
%
% \section{Documentation}
%
% Lua\TeX\ provides an extremely interesting feature, named callbacks. It
% allows to call some lua functions at some points of the \TeX\ algorithm (a
% \emph{callback}), like when \TeX\ breaks likes, puts vertical spaces, etc.
% The Lua\TeX\ core offers a function called \texttt{callback.register} that
% enables to register a function in a callback.
%
% The problem with \texttt{callback.register} is that is registers only one
% function in a callback. For a lot of callbacks it can be common to have
% several packages registering their function in a callback, and thus it is
% impossible with them to be compatible with each other.
%
% This package solves this problem by adding mainly one new function
% \texttt{callback.\\add} that adds a function in a callback. With this
% function it is possible for packages to register their function in a
% callback without overwriting the functions of the other packages.
%
% The functions are called in a certain order, and when a package registers a
% callback it can assign a priority to its function. Conflicts can still
% remain even with the priority mechanism, for example in the case where two
% packages want to have the highest priority. In these cases the packages have
% to solve the conflicts themselves.
%
% This package also privides a way to create and call new callbacks, in
% addition to the default Lua\TeX\ callbacks.
%
% This package contains only a \texttt{.lua} file, that can be called by
% another lua script.
%
%\subsubsection*{Limitations}
%
% This package only works for callbacks where it's safe to add multiple
% functions without changing the functions' signatures. There are callbacks,
% though, where registering several functions is not possible without changing
% the function's signatures, like for example the readers callbacks. These
% callbacks take a filename and give the datas in it. One solution would be to
% change the functions' signature to open it when the function is the first,
% and to take the datas and modify them eventually if they are called after
% the first. But it seems rather fragile and useless, so it's not implemented.
% With these callbacks, in this package we simply execute the first function
% in the list.
%
% Other callbacks in this case are \texttt{define\_font} and
% \texttt{open\_read\_file}. There is though a solution for several packages
% to use these callbacks, see the implementation of \texttt{luatextra}.
%
%    \section{Implementation}
%
%    \subsection{Lua module}
%
%    Module identification.
%
%    \begin{macrocode}
%<*lua>
module('luatexbase.mcb', package.seeall)
luatexbase.provides_module({
    name          = "luamcallbacks",
    version       = 0.93,
    date          = "2009/09/18",
    description   = "register several functions in a callback",
    author        = "Hans Hagen, Elie Roux and Manuel Pégourie-Gonnard",
    copyright     = "Hans Hagen, Elie Roux and Manuel Pégourie-Gonnard",
    license       = "CC0",
})
%    \end{macrocode}
%
%    \texttt{callbacklist} is the main list, that contains the callbacks as
%    keys and a table of the registered functions a values.
%
%    \begin{macrocode}

callbacklist = callbacklist or { }

%    \end{macrocode}
%
%    A table with the default functions of the created callbacks. See
%    \texttt{create} for further informations.
%
%    \begin{macrocode}

lua_callbacks_defaults = { }

local format = string.format

%    \end{macrocode}
%
%    There are 4 types of callback:
%    \begin{itemize}
%    \item the ones taking a list of nodes and returning a boolean and
%    eventually a new head (\texttt{list})
%    \item the ones taking datas and returning the modified ones
%    (\texttt{data})
%    \item the ones that can't have multiple functions registered in them
%    (\texttt{first})
%    \item the ones for functions that don't return anything (\texttt{simple})
%    \end{itemize}
%
%    \begin{macrocode}

local list = 1
local data = 2
local first = 3
local simple = 4

%    \end{macrocode}
%
%    \texttt{callbacktypes} is the list that contains the callbacks as keys
%    and the type (list or data) as values.
%
%    \begin{macrocode}

callbacktypes = callbacktypes or {
buildpage_filter = simple,
token_filter = first,
pre_output_filter = list,
hpack_filter = list,
process_input_buffer = data,
mlist_to_hlist = list,
vpack_filter = list,
define_font = first,
open_read_file = first,
linebreak_filter = list,
post_linebreak_filter = list,
pre_linebreak_filter = list,
start_page_number = simple,
stop_page_number = simple,
start_run = simple,
show_error_hook = simple,
stop_run = simple,
hyphenate = simple,
ligaturing = simple,
kerning = data,
find_write_file = first,
find_read_file = first,
find_vf_file = data,
find_map_file = data,
find_format_file = data,
find_opentype_file = data,
find_output_file = data,
find_truetype_file = data,
find_type1_file = data,
find_data_file = data,
find_pk_file = data,
find_font_file = data,
find_image_file = data,
find_ocp_file = data,
find_sfd_file = data,
find_enc_file = data,
read_sfd_file = first,
read_map_file = first,
read_pk_file = first,
read_enc_file = first,
read_vf_file = first,
read_ocp_file = first,
read_opentype_file = first,
read_truetype_file = first,
read_font_file = first,
read_type1_file = first,
read_data_file = first,
}

%    \end{macrocode}
%
%    In Lua\TeX\ version 0.43, a new callback called |process_output_buffer|
%    appeared, so we enable it.
%
%    \begin{macrocode}

if tex.luatexversion > 42 then
    callbacktypes["process_output_buffer"] = data
end

%    \end{macrocode}
%
%    As we overwrite \texttt{callback.register}, we save it as
%    \texttt{internalregister}. After that we declare some
%    functions to write the errors or the logs.
%
%    \begin{macrocode}

internalregister = internalregister or callback.register

local callbacktypes = callbacktypes

log = log or function(...)
  luatexbase.module_log('luamcallbacks', format(...))
end

info = info or function(...)
  luatexbase.module_info('luamcallbacks', format(...))
end

warning = warning or function(...)
  luatexbase.module_warning('luamcallbacks', format(...))
end

error = error or function(...)
  luatexbase.module_error('luamcallbacks', format(...))
end

%    \end{macrocode}
%
%    A simple function we'll use later to understand the arguments of the
%    \texttt{create} function. It takes a string and returns the type
%    corresponding to the string or nil.
%
%    \begin{macrocode}

function str_to_type(str)
    if str == 'list' then
        return list
    elseif str == 'data' then
        return data
    elseif str == 'first' then
        return first
    elseif str == 'simple' then
        return simple
    else
        return nil
    end
end

%    \end{macrocode}
%
%    \begin{macro}{create}
%
%    This first function creates a new callback. The signature is
%    \texttt{create(name, ctype, default)} where \texttt{name} is the name of
%    the new callback to create, \texttt{ctype} is the type of callback, and
%    \texttt{default} is the default function to call if no function is
%    registered in this callback.
%
%    The created callback will behave the same way Lua\TeX\ callbacks do, you
%    can add and remove functions in it. The difference is that the callback
%    is not automatically called, the package developer creating a new
%    callback must also call it, see next function.
%
%    \begin{macrocode}

function create(name, ctype, default)
    if not name then
        error(format("unable to call callback, no proper name passed", name))
        return nil
    end
    if not ctype or not default then
        error(format("unable to create callback '%s', callbacktype or default function not specified", name))
        return nil
    end
    if callbacktypes[name] then
        error(format("unable to create callback '%s', callback already exists", name))
        return nil
    end
    local temp = str_to_type(ctype)
    if not temp then
        error(format("unable to create callback '%s', type '%s' undefined", name, ctype))
        return nil
    end
    ctype = temp
    lua_callbacks_defaults[name] = default
    callbacktypes[name] = ctype
end

%    \end{macrocode}
%
%    \end{macro}
%
%    \begin{macro}{call}
%
%    This function calls a callback. It can only call a callback created by
%    the \texttt{create} function.
%
%    \begin{macrocode}

function call(name, ...)
    if not name then
        error(format("unable to call callback, no proper name passed", name))
        return nil
    end
    if not lua_callbacks_defaults[name] then
        error(format("unable to call lua callback '%s', unknown callback", name))
        return nil
    end
    local l = callbacklist[name]
    local f
    if not l then
        f = lua_callbacks_defaults[name]
    else
        if callbacktypes[name] == list then
            f = listhandler(name)
        elseif callbacktypes[name] == data then
            f = datahandler(name)
        elseif callbacktypes[name] == simple then
            f = simplehandler(name)
        elseif callbacktypes[name] == first then
            f = firsthandler(name)
        else
            error("unknown callback type")
        end
    end
    return f(...)
end

%    \end{macrocode}
%
%    \end{macro}
%
%    \begin{macro}{add}
%
%    The main function. The signature is \texttt{add (name,
%    func, description, priority)} with \texttt{name} being the name of the
%    callback in which the function is added; \texttt{func} is the added
%    function; \texttt{description} is a small character string describing the
%    function, and \texttt{priority} an optional argument describing the
%    priority the function will have.
%
%    The functions for a callbacks are added in a list (in
%    \texttt{callbacklist\\.callbackname}). If they have no
%    priority or a high priority number, they will be added at the end of the
%    list, and will be called after the others. If they have a low priority
%    number, the will be added at the beginning of the list and will be called
%    before the others.
%
%    Something that must be made clear, is that there is absolutely no
%    solution for packages conflicts: if two packages want the top priority on
%    a certain callback, they will have to decide the priority they will give
%    to their function themself. Most of the time, the priority is not needed.
%
%    \begin{macrocode}

function add (name,func,description,priority)
    if type(func) ~= "function" then
        error("unable to add function, no proper function passed")
        return
    end
    if not name or name == "" then
        error("unable to add function, no proper callback name passed")
        return
    elseif not callbacktypes[name] then
        error(
          format("unable to add function, '%s' is not a valid callback",
          name))
        return
    end
    if not description or description == "" then
        error(
          format("unable to add function to '%s', no proper description passed",
          name))
        return
    end
    if get_priority(name, description) ~= 0 then
        warning(
          format("function '%s' already registered in callback '%s'",
          description, name))
    end
    local l = callbacklist[name]
    if not l then
        l = {}
        callbacklist[name] = l
        if not lua_callbacks_defaults[name] then
            if callbacktypes[name] == list then
                internalregister(name, listhandler(name))
            elseif callbacktypes[name] == data then
                internalregister(name, datahandler(name))
            elseif callbacktypes[name] == simple then
                internalregister(name, simplehandler(name))
            elseif callbacktypes[name] == first then
                internalregister(name, firsthandler(name))
            else
                error("unknown callback type")
            end
        end
    end
    local f = {
        func = func,
        description = description,
    }
    priority = tonumber(priority)
    if not priority or priority > #l then
        priority = #l+1
    elseif priority < 1 then
        priority = 1
    end
    if callbacktypes[name] == first and (priority ~= 1 or #l ~= 0) then
        warning(format("several callbacks registered in callback '%s', only the first function will be active.", name))
    end
    table.insert(l,priority,f)
    log(
      format("inserting function '%s' at position %s in callback list for '%s'",
      description,priority,name))
end

%    \end{macrocode}
%
%    \end{macro}
%
%    \begin{macro}{get priority}
%
%    This function tells if a function has already been registered in a
%    callback, and gives its current priority. The arguments are the name of
%    the callback and the description of the function. If it has already been
%    registered, it gives its priority, and if not it returns false.
%
%    \begin{macrocode}

function get_priority (name, description)
    if not name or name == "" or not callbacktypes[name] or not description then
        return 0
    end
    local l = callbacklist[name]
    if not l then return 0 end
    for p, f in pairs(l) do
        if f.description == description then
            return p
        end
    end
    return 0
end

%    \end{macrocode}
%
%    \end{macro}
%
%    \begin{macro}{remove}
%
%    The function that removes a function from a callback. The signature is
%    \texttt{mcallbacks.remove (name, description)} with \texttt{name} being
%    the name of callbacks, and description the description passed to
%    \texttt{mcallbacks.add}.
%
%    \begin{macrocode}

function remove (name, description)
    if not name or name == "" then
        error("unable to remove function, no proper callback name passed")
        return
    elseif not callbacktypes[name] then
        error(
          format("unable to remove function, '%s' is not a valid callback",
          name))
        return
    end
    if not description or description == "" then
        error(
          format("unable to remove function from '%s', no proper description passed",
          name))
        return
    end
    local l = callbacklist[name]
    if not l then
        error(format("no callback list for '%s'",name))
        return
    end
    for k,v in ipairs(l) do
        if v.description == description then
            table.remove(l,k)
            log(
              format("removing function '%s' from '%s'",description,name))
            if not next(l) then
              callbacklist[name] = nil
              if not lua_callbacks_defaults[name] then
                internalregister(name, nil)
              end
            end
            return
        end
    end
    warning(
      format("unable to remove function '%s' from '%s'",description,name))
end

%    \end{macrocode}
%
%    \end{macro}
%
%    \begin{macro}{reset}
%
%    This function removes all the functions registered in a callback.
%
%    \begin{macrocode}

function reset (name)
    if not name or name == "" then
        error("unable to reset, no proper callback name passed")
        return
    elseif not callbacktypes[name] then
        error(
          format("reset error, '%s' is not a valid callback",
          name))
        return
    end
    if not lua_callbacks_defaults[name] then
        internalregister(name, nil)
    end
    local l = callbacklist[name]
    if l then
        log(format("resetting callback list '%s'",name))
        callbacklist[name] = nil
    end
end

%    \end{macrocode}
%
%    \end{macro}
%
%     This function and the following ones are only internal. This one is the
%     handler for the first type of callbacks: the ones that take a list head
%     and return true, false, or a new list head.
%
%    \begin{macro}{listhandler}
%
%    \begin{macrocode}

function listhandler (name)
    return function(head,...)
        local l = callbacklist[name]
        if l then
            local done = true
            for _, f in ipairs(l) do
                -- the returned value can be either true or a new head plus true
                rtv1, rtv2 = f.func(head,...)
                if type(rtv1) == 'boolean' then
                    done = rtv1
                elseif type (rtv1) == 'userdata' then
                    head = rtv1
                end
                if type(rtv2) == 'boolean'  then
                    done = rtv2
                elseif type(rtv2) == 'userdata' then
                    head = rtv2
                end
                if done == false then
                    error(format(
                      "function \"%s\" returned false in callback '%s'", 
                      f.description, name))
                end
            end
            return head, done
        else
            return head, false
        end
    end
end

%    \end{macrocode}
%
%    \end{macro}
%
%     The handler for callbacks taking datas and returning modified ones.
%
%    \begin{macro}{datahandler}
%
%    \begin{macrocode}

function datahandler (name)
    return function(data,...)
        local l = callbacklist[name]
        if l then
            for _, f in ipairs(l) do
                data = f.func(data,...)
            end
        end
        return data
    end
end

%    \end{macrocode}
%
%    \end{macro}
%
%     This function is for the handlers that don't support more than one
%     functions in them. In this case we only call the first function of the
%     list.
%
%    \begin{macro}{firsthandler}
%
%    \begin{macrocode}

function firsthandler (name)
    return function(...)
        local l = callbacklist[name]
        if l then
            local f = l[1].func
            return f(...)
        else
            return nil, false
        end
    end
end

%    \end{macrocode}
%
%    \end{macro}
%
%     Handler for simple functions that don't return anything.
%
%    \begin{macro}{simplehandler}
%
%    \begin{macrocode}

function simplehandler (name)
    return function(...)
        local l = callbacklist[name]
        if l then
            for _, f in ipairs(l) do
                f.func(...)
            end
        end
    end
end

%    \end{macrocode}
%
%    \end{macro}
%
%    Finally we add some functions to the \texttt{callback} module, and we
%    overwrite \texttt{callback.register} so that it outputs an error.
%
%    \begin{macrocode}

callback.add = add
callback.remove = remove
callback.reset = reset
callback.create = create
callback.call = call
callback.get_priority = get_priority

callback.register = function (...)
error("function callback.register has been deleted by luamcallbacks, please use callback.add instead.")
end

%    \end{macrocode}
%
% \iffalse
%</lua>
% \fi
%
%    \section{Test files}
%
%    A few basic tests for Plain and LaTeX.
%
%    \begin{macrocode}
%<testplain>\input luatexbase-modutils.sty
%<testlatex>\RequirePackage{luatexbase-modutils}
%<*testplain,testlatex>
\catcode 64 11
\luatexbase@directlua{
  require "luatexbase.mcb"
  local function one(head,...)
      texio.write_nl("I'm number 1")
      return head, true
  end

  local function two(head,...)
      texio.write_nl("I'm number 2")
      return head, true
  end

  local function three(head,...)
      texio.write_nl("I'm number 3")
      return head, true
  end

  callback.add("hpack_filter",one,"my example function one",1)
  callback.add("hpack_filter",two,"my example function two",2)
  callback.add("hpack_filter",three,"my example function three",1)

  callback.remove("hpack_filter","my example function three")
}
%</testplain,testlatex>
%<testplain>\bye
%<testlatex>\stop
%    \end{macrocode}
%
% \Finale
\endinput
